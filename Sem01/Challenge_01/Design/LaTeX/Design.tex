
\documentclass[a4paper,12pt]{article}

% Pakketten voor Nederlandse taal, lijsten en diagrammen
\usepackage[dutch]{babel}
\usepackage[utf8]{inputenc}
\usepackage[T1]{fontenc}
\usepackage{enumitem} % Voor betere lijsten

\title{Design v2}
\author{Bram Wieringa}
\date{\today}

\begin{document}

\maketitle

\section{Inleiding}
Als ik nu opnieuw zou beginnen, zou ik mijn designproces anders structureren met meer nadruk op de volgorde en samenhang van de diagrammen. Het starten met het uitwerken van de interviews vind ik nog steeds een waardevolle eerste stap. Tijdens deze fase veranderde mijn oorspronkelijke idee namelijk sterk: van een platform dat het gewicht meet waarop je staat, naar een halterschijf met geïntegreerde sensoren die de positie registreren.

\section{Opbouw nieuwe aanpak}

\subsection{Interviews en analyse}
\begin{enumerate}[label=\arabic*.]
    \item Interview 1 – uitwerking
    \item Interview 1 – reflectie
    \item Interview x – uitwerking
    \item Interview x – reflectie
    \item Bevindingen interviews
\end{enumerate}

\subsection{Conceptueel ontwerp}
\begin{itemize}
    \item Chen-notatie
\end{itemize}

\subsection{Logisch ontwerp}
\begin{itemize}
    \item Database-diagram
    \item Klassediagram
\end{itemize}

\subsection{Dynamisch ontwerp}
\begin{itemize}
    \item State-diagram Arduino
    \item State-diagram PC-software (optioneel)
    \item Eventueel flowcharts
\end{itemize}

\end{document}
